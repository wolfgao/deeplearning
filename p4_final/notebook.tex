
% Default to the notebook output style

    


% Inherit from the specified cell style.




    
\documentclass[11pt]{article}

    
    
    \usepackage[T1]{fontenc}
    % Nicer default font (+ math font) than Computer Modern for most use cases
    \usepackage{mathpazo}

    % Basic figure setup, for now with no caption control since it's done
    % automatically by Pandoc (which extracts ![](path) syntax from Markdown).
    \usepackage{graphicx}
    % We will generate all images so they have a width \maxwidth. This means
    % that they will get their normal width if they fit onto the page, but
    % are scaled down if they would overflow the margins.
    \makeatletter
    \def\maxwidth{\ifdim\Gin@nat@width>\linewidth\linewidth
    \else\Gin@nat@width\fi}
    \makeatother
    \let\Oldincludegraphics\includegraphics
    % Set max figure width to be 80% of text width, for now hardcoded.
    \renewcommand{\includegraphics}[1]{\Oldincludegraphics[width=.8\maxwidth]{#1}}
    % Ensure that by default, figures have no caption (until we provide a
    % proper Figure object with a Caption API and a way to capture that
    % in the conversion process - todo).
    \usepackage{caption}
    \DeclareCaptionLabelFormat{nolabel}{}
    \captionsetup{labelformat=nolabel}

    \usepackage{adjustbox} % Used to constrain images to a maximum size 
    \usepackage{xcolor} % Allow colors to be defined
    \usepackage{enumerate} % Needed for markdown enumerations to work
    \usepackage{geometry} % Used to adjust the document margins
    \usepackage{amsmath} % Equations
    \usepackage{amssymb} % Equations
    \usepackage{textcomp} % defines textquotesingle
    % Hack from http://tex.stackexchange.com/a/47451/13684:
    \AtBeginDocument{%
        \def\PYZsq{\textquotesingle}% Upright quotes in Pygmentized code
    }
    \usepackage{upquote} % Upright quotes for verbatim code
    \usepackage{eurosym} % defines \euro
    \usepackage[mathletters]{ucs} % Extended unicode (utf-8) support
    \usepackage[utf8x]{inputenc} % Allow utf-8 characters in the tex document
    \usepackage{fancyvrb} % verbatim replacement that allows latex
    \usepackage{grffile} % extends the file name processing of package graphics 
                         % to support a larger range 
    % The hyperref package gives us a pdf with properly built
    % internal navigation ('pdf bookmarks' for the table of contents,
    % internal cross-reference links, web links for URLs, etc.)
    \usepackage{hyperref}
    \usepackage{longtable} % longtable support required by pandoc >1.10
    \usepackage{booktabs}  % table support for pandoc > 1.12.2
    \usepackage[inline]{enumitem} % IRkernel/repr support (it uses the enumerate* environment)
    \usepackage[normalem]{ulem} % ulem is needed to support strikethroughs (\sout)
                                % normalem makes italics be italics, not underlines
    

    
    
    % Colors for the hyperref package
    \definecolor{urlcolor}{rgb}{0,.145,.698}
    \definecolor{linkcolor}{rgb}{.71,0.21,0.01}
    \definecolor{citecolor}{rgb}{.12,.54,.11}

    % ANSI colors
    \definecolor{ansi-black}{HTML}{3E424D}
    \definecolor{ansi-black-intense}{HTML}{282C36}
    \definecolor{ansi-red}{HTML}{E75C58}
    \definecolor{ansi-red-intense}{HTML}{B22B31}
    \definecolor{ansi-green}{HTML}{00A250}
    \definecolor{ansi-green-intense}{HTML}{007427}
    \definecolor{ansi-yellow}{HTML}{DDB62B}
    \definecolor{ansi-yellow-intense}{HTML}{B27D12}
    \definecolor{ansi-blue}{HTML}{208FFB}
    \definecolor{ansi-blue-intense}{HTML}{0065CA}
    \definecolor{ansi-magenta}{HTML}{D160C4}
    \definecolor{ansi-magenta-intense}{HTML}{A03196}
    \definecolor{ansi-cyan}{HTML}{60C6C8}
    \definecolor{ansi-cyan-intense}{HTML}{258F8F}
    \definecolor{ansi-white}{HTML}{C5C1B4}
    \definecolor{ansi-white-intense}{HTML}{A1A6B2}

    % commands and environments needed by pandoc snippets
    % extracted from the output of `pandoc -s`
    \providecommand{\tightlist}{%
      \setlength{\itemsep}{0pt}\setlength{\parskip}{0pt}}
    \DefineVerbatimEnvironment{Highlighting}{Verbatim}{commandchars=\\\{\}}
    % Add ',fontsize=\small' for more characters per line
    \newenvironment{Shaded}{}{}
    \newcommand{\KeywordTok}[1]{\textcolor[rgb]{0.00,0.44,0.13}{\textbf{{#1}}}}
    \newcommand{\DataTypeTok}[1]{\textcolor[rgb]{0.56,0.13,0.00}{{#1}}}
    \newcommand{\DecValTok}[1]{\textcolor[rgb]{0.25,0.63,0.44}{{#1}}}
    \newcommand{\BaseNTok}[1]{\textcolor[rgb]{0.25,0.63,0.44}{{#1}}}
    \newcommand{\FloatTok}[1]{\textcolor[rgb]{0.25,0.63,0.44}{{#1}}}
    \newcommand{\CharTok}[1]{\textcolor[rgb]{0.25,0.44,0.63}{{#1}}}
    \newcommand{\StringTok}[1]{\textcolor[rgb]{0.25,0.44,0.63}{{#1}}}
    \newcommand{\CommentTok}[1]{\textcolor[rgb]{0.38,0.63,0.69}{\textit{{#1}}}}
    \newcommand{\OtherTok}[1]{\textcolor[rgb]{0.00,0.44,0.13}{{#1}}}
    \newcommand{\AlertTok}[1]{\textcolor[rgb]{1.00,0.00,0.00}{\textbf{{#1}}}}
    \newcommand{\FunctionTok}[1]{\textcolor[rgb]{0.02,0.16,0.49}{{#1}}}
    \newcommand{\RegionMarkerTok}[1]{{#1}}
    \newcommand{\ErrorTok}[1]{\textcolor[rgb]{1.00,0.00,0.00}{\textbf{{#1}}}}
    \newcommand{\NormalTok}[1]{{#1}}
    
    % Additional commands for more recent versions of Pandoc
    \newcommand{\ConstantTok}[1]{\textcolor[rgb]{0.53,0.00,0.00}{{#1}}}
    \newcommand{\SpecialCharTok}[1]{\textcolor[rgb]{0.25,0.44,0.63}{{#1}}}
    \newcommand{\VerbatimStringTok}[1]{\textcolor[rgb]{0.25,0.44,0.63}{{#1}}}
    \newcommand{\SpecialStringTok}[1]{\textcolor[rgb]{0.73,0.40,0.53}{{#1}}}
    \newcommand{\ImportTok}[1]{{#1}}
    \newcommand{\DocumentationTok}[1]{\textcolor[rgb]{0.73,0.13,0.13}{\textit{{#1}}}}
    \newcommand{\AnnotationTok}[1]{\textcolor[rgb]{0.38,0.63,0.69}{\textbf{\textit{{#1}}}}}
    \newcommand{\CommentVarTok}[1]{\textcolor[rgb]{0.38,0.63,0.69}{\textbf{\textit{{#1}}}}}
    \newcommand{\VariableTok}[1]{\textcolor[rgb]{0.10,0.09,0.49}{{#1}}}
    \newcommand{\ControlFlowTok}[1]{\textcolor[rgb]{0.00,0.44,0.13}{\textbf{{#1}}}}
    \newcommand{\OperatorTok}[1]{\textcolor[rgb]{0.40,0.40,0.40}{{#1}}}
    \newcommand{\BuiltInTok}[1]{{#1}}
    \newcommand{\ExtensionTok}[1]{{#1}}
    \newcommand{\PreprocessorTok}[1]{\textcolor[rgb]{0.74,0.48,0.00}{{#1}}}
    \newcommand{\AttributeTok}[1]{\textcolor[rgb]{0.49,0.56,0.16}{{#1}}}
    \newcommand{\InformationTok}[1]{\textcolor[rgb]{0.38,0.63,0.69}{\textbf{\textit{{#1}}}}}
    \newcommand{\WarningTok}[1]{\textcolor[rgb]{0.38,0.63,0.69}{\textbf{\textit{{#1}}}}}
    
    
    % Define a nice break command that doesn't care if a line doesn't already
    % exist.
    \def\br{\hspace*{\fill} \\* }
    % Math Jax compatability definitions
    \def\gt{>}
    \def\lt{<}
    % Document parameters
    \title{Titanic\_passengers\_analysis}
    
    
    

    % Pygments definitions
    
\makeatletter
\def\PY@reset{\let\PY@it=\relax \let\PY@bf=\relax%
    \let\PY@ul=\relax \let\PY@tc=\relax%
    \let\PY@bc=\relax \let\PY@ff=\relax}
\def\PY@tok#1{\csname PY@tok@#1\endcsname}
\def\PY@toks#1+{\ifx\relax#1\empty\else%
    \PY@tok{#1}\expandafter\PY@toks\fi}
\def\PY@do#1{\PY@bc{\PY@tc{\PY@ul{%
    \PY@it{\PY@bf{\PY@ff{#1}}}}}}}
\def\PY#1#2{\PY@reset\PY@toks#1+\relax+\PY@do{#2}}

\expandafter\def\csname PY@tok@gd\endcsname{\def\PY@tc##1{\textcolor[rgb]{0.63,0.00,0.00}{##1}}}
\expandafter\def\csname PY@tok@gu\endcsname{\let\PY@bf=\textbf\def\PY@tc##1{\textcolor[rgb]{0.50,0.00,0.50}{##1}}}
\expandafter\def\csname PY@tok@gt\endcsname{\def\PY@tc##1{\textcolor[rgb]{0.00,0.27,0.87}{##1}}}
\expandafter\def\csname PY@tok@gs\endcsname{\let\PY@bf=\textbf}
\expandafter\def\csname PY@tok@gr\endcsname{\def\PY@tc##1{\textcolor[rgb]{1.00,0.00,0.00}{##1}}}
\expandafter\def\csname PY@tok@cm\endcsname{\let\PY@it=\textit\def\PY@tc##1{\textcolor[rgb]{0.25,0.50,0.50}{##1}}}
\expandafter\def\csname PY@tok@vg\endcsname{\def\PY@tc##1{\textcolor[rgb]{0.10,0.09,0.49}{##1}}}
\expandafter\def\csname PY@tok@vi\endcsname{\def\PY@tc##1{\textcolor[rgb]{0.10,0.09,0.49}{##1}}}
\expandafter\def\csname PY@tok@vm\endcsname{\def\PY@tc##1{\textcolor[rgb]{0.10,0.09,0.49}{##1}}}
\expandafter\def\csname PY@tok@mh\endcsname{\def\PY@tc##1{\textcolor[rgb]{0.40,0.40,0.40}{##1}}}
\expandafter\def\csname PY@tok@cs\endcsname{\let\PY@it=\textit\def\PY@tc##1{\textcolor[rgb]{0.25,0.50,0.50}{##1}}}
\expandafter\def\csname PY@tok@ge\endcsname{\let\PY@it=\textit}
\expandafter\def\csname PY@tok@vc\endcsname{\def\PY@tc##1{\textcolor[rgb]{0.10,0.09,0.49}{##1}}}
\expandafter\def\csname PY@tok@il\endcsname{\def\PY@tc##1{\textcolor[rgb]{0.40,0.40,0.40}{##1}}}
\expandafter\def\csname PY@tok@go\endcsname{\def\PY@tc##1{\textcolor[rgb]{0.53,0.53,0.53}{##1}}}
\expandafter\def\csname PY@tok@cp\endcsname{\def\PY@tc##1{\textcolor[rgb]{0.74,0.48,0.00}{##1}}}
\expandafter\def\csname PY@tok@gi\endcsname{\def\PY@tc##1{\textcolor[rgb]{0.00,0.63,0.00}{##1}}}
\expandafter\def\csname PY@tok@gh\endcsname{\let\PY@bf=\textbf\def\PY@tc##1{\textcolor[rgb]{0.00,0.00,0.50}{##1}}}
\expandafter\def\csname PY@tok@ni\endcsname{\let\PY@bf=\textbf\def\PY@tc##1{\textcolor[rgb]{0.60,0.60,0.60}{##1}}}
\expandafter\def\csname PY@tok@nl\endcsname{\def\PY@tc##1{\textcolor[rgb]{0.63,0.63,0.00}{##1}}}
\expandafter\def\csname PY@tok@nn\endcsname{\let\PY@bf=\textbf\def\PY@tc##1{\textcolor[rgb]{0.00,0.00,1.00}{##1}}}
\expandafter\def\csname PY@tok@no\endcsname{\def\PY@tc##1{\textcolor[rgb]{0.53,0.00,0.00}{##1}}}
\expandafter\def\csname PY@tok@na\endcsname{\def\PY@tc##1{\textcolor[rgb]{0.49,0.56,0.16}{##1}}}
\expandafter\def\csname PY@tok@nb\endcsname{\def\PY@tc##1{\textcolor[rgb]{0.00,0.50,0.00}{##1}}}
\expandafter\def\csname PY@tok@nc\endcsname{\let\PY@bf=\textbf\def\PY@tc##1{\textcolor[rgb]{0.00,0.00,1.00}{##1}}}
\expandafter\def\csname PY@tok@nd\endcsname{\def\PY@tc##1{\textcolor[rgb]{0.67,0.13,1.00}{##1}}}
\expandafter\def\csname PY@tok@ne\endcsname{\let\PY@bf=\textbf\def\PY@tc##1{\textcolor[rgb]{0.82,0.25,0.23}{##1}}}
\expandafter\def\csname PY@tok@nf\endcsname{\def\PY@tc##1{\textcolor[rgb]{0.00,0.00,1.00}{##1}}}
\expandafter\def\csname PY@tok@si\endcsname{\let\PY@bf=\textbf\def\PY@tc##1{\textcolor[rgb]{0.73,0.40,0.53}{##1}}}
\expandafter\def\csname PY@tok@s2\endcsname{\def\PY@tc##1{\textcolor[rgb]{0.73,0.13,0.13}{##1}}}
\expandafter\def\csname PY@tok@nt\endcsname{\let\PY@bf=\textbf\def\PY@tc##1{\textcolor[rgb]{0.00,0.50,0.00}{##1}}}
\expandafter\def\csname PY@tok@nv\endcsname{\def\PY@tc##1{\textcolor[rgb]{0.10,0.09,0.49}{##1}}}
\expandafter\def\csname PY@tok@s1\endcsname{\def\PY@tc##1{\textcolor[rgb]{0.73,0.13,0.13}{##1}}}
\expandafter\def\csname PY@tok@dl\endcsname{\def\PY@tc##1{\textcolor[rgb]{0.73,0.13,0.13}{##1}}}
\expandafter\def\csname PY@tok@ch\endcsname{\let\PY@it=\textit\def\PY@tc##1{\textcolor[rgb]{0.25,0.50,0.50}{##1}}}
\expandafter\def\csname PY@tok@m\endcsname{\def\PY@tc##1{\textcolor[rgb]{0.40,0.40,0.40}{##1}}}
\expandafter\def\csname PY@tok@gp\endcsname{\let\PY@bf=\textbf\def\PY@tc##1{\textcolor[rgb]{0.00,0.00,0.50}{##1}}}
\expandafter\def\csname PY@tok@sh\endcsname{\def\PY@tc##1{\textcolor[rgb]{0.73,0.13,0.13}{##1}}}
\expandafter\def\csname PY@tok@ow\endcsname{\let\PY@bf=\textbf\def\PY@tc##1{\textcolor[rgb]{0.67,0.13,1.00}{##1}}}
\expandafter\def\csname PY@tok@sx\endcsname{\def\PY@tc##1{\textcolor[rgb]{0.00,0.50,0.00}{##1}}}
\expandafter\def\csname PY@tok@bp\endcsname{\def\PY@tc##1{\textcolor[rgb]{0.00,0.50,0.00}{##1}}}
\expandafter\def\csname PY@tok@c1\endcsname{\let\PY@it=\textit\def\PY@tc##1{\textcolor[rgb]{0.25,0.50,0.50}{##1}}}
\expandafter\def\csname PY@tok@fm\endcsname{\def\PY@tc##1{\textcolor[rgb]{0.00,0.00,1.00}{##1}}}
\expandafter\def\csname PY@tok@o\endcsname{\def\PY@tc##1{\textcolor[rgb]{0.40,0.40,0.40}{##1}}}
\expandafter\def\csname PY@tok@kc\endcsname{\let\PY@bf=\textbf\def\PY@tc##1{\textcolor[rgb]{0.00,0.50,0.00}{##1}}}
\expandafter\def\csname PY@tok@c\endcsname{\let\PY@it=\textit\def\PY@tc##1{\textcolor[rgb]{0.25,0.50,0.50}{##1}}}
\expandafter\def\csname PY@tok@mf\endcsname{\def\PY@tc##1{\textcolor[rgb]{0.40,0.40,0.40}{##1}}}
\expandafter\def\csname PY@tok@err\endcsname{\def\PY@bc##1{\setlength{\fboxsep}{0pt}\fcolorbox[rgb]{1.00,0.00,0.00}{1,1,1}{\strut ##1}}}
\expandafter\def\csname PY@tok@mb\endcsname{\def\PY@tc##1{\textcolor[rgb]{0.40,0.40,0.40}{##1}}}
\expandafter\def\csname PY@tok@ss\endcsname{\def\PY@tc##1{\textcolor[rgb]{0.10,0.09,0.49}{##1}}}
\expandafter\def\csname PY@tok@sr\endcsname{\def\PY@tc##1{\textcolor[rgb]{0.73,0.40,0.53}{##1}}}
\expandafter\def\csname PY@tok@mo\endcsname{\def\PY@tc##1{\textcolor[rgb]{0.40,0.40,0.40}{##1}}}
\expandafter\def\csname PY@tok@kd\endcsname{\let\PY@bf=\textbf\def\PY@tc##1{\textcolor[rgb]{0.00,0.50,0.00}{##1}}}
\expandafter\def\csname PY@tok@mi\endcsname{\def\PY@tc##1{\textcolor[rgb]{0.40,0.40,0.40}{##1}}}
\expandafter\def\csname PY@tok@kn\endcsname{\let\PY@bf=\textbf\def\PY@tc##1{\textcolor[rgb]{0.00,0.50,0.00}{##1}}}
\expandafter\def\csname PY@tok@cpf\endcsname{\let\PY@it=\textit\def\PY@tc##1{\textcolor[rgb]{0.25,0.50,0.50}{##1}}}
\expandafter\def\csname PY@tok@kr\endcsname{\let\PY@bf=\textbf\def\PY@tc##1{\textcolor[rgb]{0.00,0.50,0.00}{##1}}}
\expandafter\def\csname PY@tok@s\endcsname{\def\PY@tc##1{\textcolor[rgb]{0.73,0.13,0.13}{##1}}}
\expandafter\def\csname PY@tok@kp\endcsname{\def\PY@tc##1{\textcolor[rgb]{0.00,0.50,0.00}{##1}}}
\expandafter\def\csname PY@tok@w\endcsname{\def\PY@tc##1{\textcolor[rgb]{0.73,0.73,0.73}{##1}}}
\expandafter\def\csname PY@tok@kt\endcsname{\def\PY@tc##1{\textcolor[rgb]{0.69,0.00,0.25}{##1}}}
\expandafter\def\csname PY@tok@sc\endcsname{\def\PY@tc##1{\textcolor[rgb]{0.73,0.13,0.13}{##1}}}
\expandafter\def\csname PY@tok@sb\endcsname{\def\PY@tc##1{\textcolor[rgb]{0.73,0.13,0.13}{##1}}}
\expandafter\def\csname PY@tok@sa\endcsname{\def\PY@tc##1{\textcolor[rgb]{0.73,0.13,0.13}{##1}}}
\expandafter\def\csname PY@tok@k\endcsname{\let\PY@bf=\textbf\def\PY@tc##1{\textcolor[rgb]{0.00,0.50,0.00}{##1}}}
\expandafter\def\csname PY@tok@se\endcsname{\let\PY@bf=\textbf\def\PY@tc##1{\textcolor[rgb]{0.73,0.40,0.13}{##1}}}
\expandafter\def\csname PY@tok@sd\endcsname{\let\PY@it=\textit\def\PY@tc##1{\textcolor[rgb]{0.73,0.13,0.13}{##1}}}

\def\PYZbs{\char`\\}
\def\PYZus{\char`\_}
\def\PYZob{\char`\{}
\def\PYZcb{\char`\}}
\def\PYZca{\char`\^}
\def\PYZam{\char`\&}
\def\PYZlt{\char`\<}
\def\PYZgt{\char`\>}
\def\PYZsh{\char`\#}
\def\PYZpc{\char`\%}
\def\PYZdl{\char`\$}
\def\PYZhy{\char`\-}
\def\PYZsq{\char`\'}
\def\PYZdq{\char`\"}
\def\PYZti{\char`\~}
% for compatibility with earlier versions
\def\PYZat{@}
\def\PYZlb{[}
\def\PYZrb{]}
\makeatother


    % Exact colors from NB
    \definecolor{incolor}{rgb}{0.0, 0.0, 0.5}
    \definecolor{outcolor}{rgb}{0.545, 0.0, 0.0}



    
    % Prevent overflowing lines due to hard-to-break entities
    \sloppy 
    % Setup hyperref package
    \hypersetup{
      breaklinks=true,  % so long urls are correctly broken across lines
      colorlinks=true,
      urlcolor=urlcolor,
      linkcolor=linkcolor,
      citecolor=citecolor,
      }
    % Slightly bigger margins than the latex defaults
    
    \geometry{verbose,tmargin=1in,bmargin=1in,lmargin=1in,rmargin=1in}
    
    

    \begin{document}
    
    
    \maketitle
    
    

    
    \subsection{建立问题和设定目标}\label{ux5efaux7acbux95eeux9898ux548cux8bbeux5b9aux76eeux6807}

首先这是一个数据分析的练习项目,我选择了泰坦尼克船客的数据进行分析。

\subsubsection{提出问题}\label{ux63d0ux51faux95eeux9898}

是什么因素导致高概率的存活?是否和性别,年龄,船舱级别或者其他任何因素有关?
如果回答这个问题,我们将从\textbf{性别,年龄,船舱级别,兄弟姐妹个数,父母或者还在在船个数来计算,最后乘客票的花费}。

    \subsection{数据加工和数据探索}\label{ux6570ux636eux52a0ux5de5ux548cux6570ux636eux63a2ux7d22}

\subsubsection{-
数据获取和数据概览}\label{ux6570ux636eux83b7ux53d6ux548cux6570ux636eux6982ux89c8}

本案例分析的所有数据来自\href{https://www.kaggle.com/c/titanic/data}{Kaggle
网站}, 该网站对数据的说明如下:

\begin{verbatim}
Variable    Definition  Key
- survival  Survival    0 = No, 1 = Yes
- pclass    Ticket class    1 = 1st, 2 = 2nd, 3 = 3rd
- sex       Sex 
- Age       Age in years    
- sibsp # of siblings / spouses aboard the Titanic  
- parch # of parents / children aboard the Titanic  
- ticket    Ticket number   
- fare  Passenger fare  
- cabin Cabin number    
- embarked  Port of Embarkation C = Cherbourg, Q = Queenstown, S = Southampton
\end{verbatim}

    \begin{Verbatim}[commandchars=\\\{\}]
{\color{incolor}In [{\color{incolor}220}]:} \PY{c+c1}{\PYZsh{}\PYZsh{}导入各种数据分析和画图模块}
          \PY{k+kn}{import} \PY{n+nn}{pandas} \PY{k+kn}{as} \PY{n+nn}{pd}
          \PY{k+kn}{import} \PY{n+nn}{numpy} \PY{k+kn}{as} \PY{n+nn}{np}
          \PY{k+kn}{import} \PY{n+nn}{matplotlib.pyplot} \PY{k+kn}{as} \PY{n+nn}{plt}
          \PY{k+kn}{import} \PY{n+nn}{seaborn} \PY{k+kn}{as} \PY{n+nn}{sns}
          
          \PY{o}{\PYZpc{}}\PY{k}{matplotlib} inline
          
          \PY{c+c1}{\PYZsh{}use read\PYZus{}csv() to get all data}
          \PY{n}{file\PYZus{}name} \PY{o}{=} \PY{l+s+s1}{\PYZsq{}}\PY{l+s+s1}{train.csv}\PY{l+s+s1}{\PYZsq{}}
          \PY{n}{titanic\PYZus{}df} \PY{o}{=} \PY{n}{pd}\PY{o}{.}\PY{n}{read\PYZus{}csv}\PY{p}{(}\PY{n}{file\PYZus{}name}\PY{p}{)}
          
          \PY{k}{print} \PY{n}{titanic\PYZus{}df}\PY{o}{.}\PY{n}{describe}\PY{p}{(}\PY{p}{)}
          \PY{k}{print} \PY{n}{titanic\PYZus{}df}\PY{o}{.}\PY{n}{head}\PY{p}{(}\PY{p}{)}
\end{Verbatim}


    \begin{Verbatim}[commandchars=\\\{\}]
       PassengerId    Survived      Pclass         Age       SibSp  \textbackslash{}
count   891.000000  891.000000  891.000000  714.000000  891.000000   
mean    446.000000    0.383838    2.308642   29.699118    0.523008   
std     257.353842    0.486592    0.836071   14.526497    1.102743   
min       1.000000    0.000000    1.000000    0.420000    0.000000   
25\%     223.500000    0.000000    2.000000   20.125000    0.000000   
50\%     446.000000    0.000000    3.000000   28.000000    0.000000   
75\%     668.500000    1.000000    3.000000   38.000000    1.000000   
max     891.000000    1.000000    3.000000   80.000000    8.000000   

            Parch        Fare  
count  891.000000  891.000000  
mean     0.381594   32.204208  
std      0.806057   49.693429  
min      0.000000    0.000000  
25\%      0.000000    7.910400  
50\%      0.000000   14.454200  
75\%      0.000000   31.000000  
max      6.000000  512.329200  
   PassengerId  Survived  Pclass  \textbackslash{}
0            1         0       3   
1            2         1       1   
2            3         1       3   
3            4         1       1   
4            5         0       3   

                                                Name     Sex   Age  SibSp  \textbackslash{}
0                            Braund, Mr. Owen Harris    male  22.0      1   
1  Cumings, Mrs. John Bradley (Florence Briggs Th{\ldots}  female  38.0      1   
2                             Heikkinen, Miss. Laina  female  26.0      0   
3       Futrelle, Mrs. Jacques Heath (Lily May Peel)  female  35.0      1   
4                           Allen, Mr. William Henry    male  35.0      0   

   Parch            Ticket     Fare Cabin Embarked  
0      0         A/5 21171   7.2500   NaN        S  
1      0          PC 17599  71.2833   C85        C  
2      0  STON/O2. 3101282   7.9250   NaN        S  
3      0            113803  53.1000  C123        S  
4      0            373450   8.0500   NaN        S  

    \end{Verbatim}

    从上面的信息看,总共有891个记录,或者说有891个乘客。另外发现Age数据不全,总共才714个记录,另外Cabin和Embarked数据也不全,我们需要想办法补充这部分数据。

    \begin{Verbatim}[commandchars=\\\{\}]
{\color{incolor}In [{\color{incolor}221}]:} \PY{k}{print} \PY{n}{titanic\PYZus{}df}\PY{o}{.}\PY{n}{groupby}\PY{p}{(}\PY{p}{[}\PY{l+s+s1}{\PYZsq{}}\PY{l+s+s1}{Survived}\PY{l+s+s1}{\PYZsq{}}\PY{p}{]}\PY{p}{)}\PY{o}{.}\PY{n}{size}\PY{p}{(}\PY{p}{)}
          \PY{n}{total\PYZus{}no\PYZus{}survived\PYZus{}num} \PY{o}{=} \PY{n}{titanic\PYZus{}df}\PY{o}{.}\PY{n}{groupby}\PY{p}{(}\PY{p}{[}\PY{l+s+s1}{\PYZsq{}}\PY{l+s+s1}{Survived}\PY{l+s+s1}{\PYZsq{}}\PY{p}{]}\PY{p}{)}\PY{o}{.}\PY{n}{size}\PY{p}{(}\PY{p}{)}\PY{p}{[}\PY{l+m+mi}{0}\PY{p}{]}
          \PY{n}{total\PYZus{}survived\PYZus{}num} \PY{o}{=} \PY{n}{titanic\PYZus{}df}\PY{o}{.}\PY{n}{groupby}\PY{p}{(}\PY{p}{[}\PY{l+s+s1}{\PYZsq{}}\PY{l+s+s1}{Survived}\PY{l+s+s1}{\PYZsq{}}\PY{p}{]}\PY{p}{)}\PY{o}{.}\PY{n}{size}\PY{p}{(}\PY{p}{)}\PY{p}{[}\PY{l+m+mi}{1}\PY{p}{]}
\end{Verbatim}


    \begin{Verbatim}[commandchars=\\\{\}]
Survived
0    549
1    342
dtype: int64

    \end{Verbatim}

    在当前这份训练数据里面,现这891个乘客到底只有342人生还,549人丧生,整体存活率是38\%,死亡率是62\%,具体哪些因素和死亡率有关,我们需要一步步来进行数据分析。

    \begin{Verbatim}[commandchars=\\\{\}]
{\color{incolor}In [{\color{incolor}222}]:} \PY{n}{plt}\PY{o}{.}\PY{n}{figure}\PY{p}{(}\PY{n}{figsize} \PY{o}{=} \PY{p}{(}\PY{l+m+mi}{5}\PY{p}{,}\PY{l+m+mi}{5}\PY{p}{)}\PY{p}{)}
          \PY{n}{plt}\PY{o}{.}\PY{n}{pie}\PY{p}{(}\PY{p}{[}\PY{n}{total\PYZus{}no\PYZus{}survived\PYZus{}num}\PY{p}{,} \PY{n}{total\PYZus{}survived\PYZus{}num}\PY{p}{]}\PY{p}{,}\PY{n}{labels}\PY{o}{=}\PY{p}{[}\PY{l+s+s1}{\PYZsq{}}\PY{l+s+s1}{No Survived}\PY{l+s+s1}{\PYZsq{}}\PY{p}{,}\PY{l+s+s1}{\PYZsq{}}\PY{l+s+s1}{Survived}\PY{l+s+s1}{\PYZsq{}}\PY{p}{]}\PY{p}{,}\PY{n}{autopct}\PY{o}{=}\PY{l+s+s1}{\PYZsq{}}\PY{l+s+si}{\PYZpc{}1.00f}\PY{l+s+si}{\PYZpc{}\PYZpc{}}\PY{l+s+s1}{\PYZsq{}}\PY{p}{)}
          \PY{n}{plt}\PY{o}{.}\PY{n}{title}\PY{p}{(}\PY{l+s+s1}{\PYZsq{}}\PY{l+s+s1}{Survival rate}\PY{l+s+s1}{\PYZsq{}}\PY{p}{)} 
          \PY{n}{plt}\PY{o}{.}\PY{n}{show}\PY{p}{(}\PY{p}{)}
\end{Verbatim}


    \begin{center}
    \adjustimage{max size={0.9\linewidth}{0.9\paperheight}}{output_6_0.png}
    \end{center}
    { \hspace*{\fill} \\}
    
    \subsubsection{- 数据探索}\label{ux6570ux636eux63a2ux7d22}

\paragraph{- 性别}\label{ux6027ux522b}

    \begin{Verbatim}[commandchars=\\\{\}]
{\color{incolor}In [{\color{incolor}223}]:} \PY{c+c1}{\PYZsh{}尝试用sns画图,分析不同性别的比例和不同性别的生还比例}
          \PY{k}{print} \PY{n}{titanic\PYZus{}df}\PY{o}{.}\PY{n}{groupby}\PY{p}{(}\PY{p}{[}\PY{l+s+s1}{\PYZsq{}}\PY{l+s+s1}{Sex}\PY{l+s+s1}{\PYZsq{}}\PY{p}{,}\PY{l+s+s1}{\PYZsq{}}\PY{l+s+s1}{Survived}\PY{l+s+s1}{\PYZsq{}}\PY{p}{]}\PY{p}{)}\PY{o}{.}\PY{n}{size}\PY{p}{(}\PY{p}{)}
          
          \PY{n}{fig}\PY{p}{,} \PY{p}{(}\PY{n}{axis1}\PY{p}{,}\PY{n}{axis2}\PY{p}{)} \PY{o}{=} \PY{n}{plt}\PY{o}{.}\PY{n}{subplots}\PY{p}{(}\PY{l+m+mi}{1}\PY{p}{,}\PY{l+m+mi}{2}\PY{p}{,}\PY{n}{figsize}\PY{o}{=}\PY{p}{(}\PY{l+m+mi}{10}\PY{p}{,}\PY{l+m+mi}{5}\PY{p}{)}\PY{p}{)} 
          \PY{n}{sns}\PY{o}{.}\PY{n}{countplot}\PY{p}{(}\PY{n}{x}\PY{o}{=}\PY{l+s+s1}{\PYZsq{}}\PY{l+s+s1}{Sex}\PY{l+s+s1}{\PYZsq{}}\PY{p}{,}\PY{n}{data}\PY{o}{=}\PY{n}{titanic\PYZus{}df}\PY{p}{,}\PY{n}{ax}\PY{o}{=}\PY{n}{axis1}\PY{p}{)}
          \PY{c+c1}{\PYZsh{}用Sex做横轴,看一下Survived的数据}
          \PY{n}{sns}\PY{o}{.}\PY{n}{countplot}\PY{p}{(}\PY{n}{x}\PY{o}{=}\PY{l+s+s1}{\PYZsq{}}\PY{l+s+s1}{Sex}\PY{l+s+s1}{\PYZsq{}}\PY{p}{,}\PY{n}{data}\PY{o}{=}\PY{n}{titanic\PYZus{}df}\PY{p}{,}\PY{n}{hue}\PY{o}{=}\PY{l+s+s2}{\PYZdq{}}\PY{l+s+s2}{Survived}\PY{l+s+s2}{\PYZdq{}}\PY{p}{,}\PY{n}{ax}\PY{o}{=}\PY{n}{axis2}\PY{p}{)}
\end{Verbatim}


    \begin{Verbatim}[commandchars=\\\{\}]
Sex     Survived
female  0            81
        1           233
male    0           468
        1           109
dtype: int64

    \end{Verbatim}

\begin{Verbatim}[commandchars=\\\{\}]
{\color{outcolor}Out[{\color{outcolor}223}]:} <matplotlib.axes.\_subplots.AxesSubplot at 0x1a1bf7a090>
\end{Verbatim}
            
    \begin{center}
    \adjustimage{max size={0.9\linewidth}{0.9\paperheight}}{output_8_2.png}
    \end{center}
    { \hspace*{\fill} \\}
    
    从上图我们可以明显看出男性存活的总数几乎死亡的1/4,而女性存活的数目差不多是死亡人数的一倍多。这个反映了当时的社会还是非常文明,当遇到危险或者灾难的时候,男人们比较绅士,女人先走,先救女人,女人获得了比男性几倍的生还几率。
我们再看看到底男性和女性的生还比例到底是多少?

    \begin{Verbatim}[commandchars=\\\{\}]
{\color{incolor}In [{\color{incolor}224}]:} \PY{c+c1}{\PYZsh{}女性的生还比例}
          \PY{n}{total\PYZus{}female\PYZus{}survived} \PY{o}{=} \PY{n}{titanic\PYZus{}df}\PY{o}{.}\PY{n}{groupby}\PY{p}{(}\PY{p}{[}\PY{l+s+s1}{\PYZsq{}}\PY{l+s+s1}{Sex}\PY{l+s+s1}{\PYZsq{}}\PY{p}{,}\PY{l+s+s1}{\PYZsq{}}\PY{l+s+s1}{Survived}\PY{l+s+s1}{\PYZsq{}}\PY{p}{]}\PY{p}{)}\PY{o}{.}\PY{n}{size}\PY{p}{(}\PY{p}{)}\PY{p}{[}\PY{l+m+mi}{1}\PY{p}{]}
          \PY{n}{total\PYZus{}female\PYZus{}dead} \PY{o}{=} \PY{n}{titanic\PYZus{}df}\PY{o}{.}\PY{n}{groupby}\PY{p}{(}\PY{p}{[}\PY{l+s+s1}{\PYZsq{}}\PY{l+s+s1}{Sex}\PY{l+s+s1}{\PYZsq{}}\PY{p}{,}\PY{l+s+s1}{\PYZsq{}}\PY{l+s+s1}{Survived}\PY{l+s+s1}{\PYZsq{}}\PY{p}{]}\PY{p}{)}\PY{o}{.}\PY{n}{size}\PY{p}{(}\PY{p}{)}\PY{p}{[}\PY{l+m+mi}{0}\PY{p}{]}
          \PY{n}{female\PYZus{}survived\PYZus{}rate} \PY{o}{=} \PY{n+nb}{float}\PY{p}{(}\PY{n}{total\PYZus{}female\PYZus{}survived}\PY{p}{)}\PY{o}{/}\PY{p}{(}\PY{n}{total\PYZus{}female\PYZus{}dead}\PY{o}{+}\PY{n}{total\PYZus{}female\PYZus{}survived}\PY{p}{)}
          \PY{k}{print} \PY{l+s+s2}{\PYZdq{}}\PY{l+s+s2}{女性的生还比例:}\PY{l+s+si}{\PYZpc{}f}\PY{l+s+s2}{\PYZdq{}} \PY{o}{\PYZpc{}}\PY{k}{female\PYZus{}survived\PYZus{}rate}
          
          \PY{c+c1}{\PYZsh{}男性的生还比例}
          \PY{n}{total\PYZus{}male\PYZus{}survived} \PY{o}{=} \PY{n}{titanic\PYZus{}df}\PY{o}{.}\PY{n}{groupby}\PY{p}{(}\PY{p}{[}\PY{l+s+s1}{\PYZsq{}}\PY{l+s+s1}{Sex}\PY{l+s+s1}{\PYZsq{}}\PY{p}{,}\PY{l+s+s1}{\PYZsq{}}\PY{l+s+s1}{Survived}\PY{l+s+s1}{\PYZsq{}}\PY{p}{]}\PY{p}{)}\PY{o}{.}\PY{n}{size}\PY{p}{(}\PY{p}{)}\PY{p}{[}\PY{l+m+mi}{3}\PY{p}{]}
          \PY{n}{total\PYZus{}male\PYZus{}dead} \PY{o}{=} \PY{n}{titanic\PYZus{}df}\PY{o}{.}\PY{n}{groupby}\PY{p}{(}\PY{p}{[}\PY{l+s+s1}{\PYZsq{}}\PY{l+s+s1}{Sex}\PY{l+s+s1}{\PYZsq{}}\PY{p}{,}\PY{l+s+s1}{\PYZsq{}}\PY{l+s+s1}{Survived}\PY{l+s+s1}{\PYZsq{}}\PY{p}{]}\PY{p}{)}\PY{o}{.}\PY{n}{size}\PY{p}{(}\PY{p}{)}\PY{p}{[}\PY{l+m+mi}{2}\PY{p}{]}
          \PY{n}{male\PYZus{}survived\PYZus{}rate} \PY{o}{=} \PY{n+nb}{float}\PY{p}{(}\PY{n}{total\PYZus{}male\PYZus{}survived}\PY{p}{)}\PY{o}{/}\PY{p}{(}\PY{n}{total\PYZus{}male\PYZus{}dead}\PY{o}{+}\PY{n}{total\PYZus{}male\PYZus{}survived}\PY{p}{)}
          \PY{k}{print} \PY{l+s+s2}{\PYZdq{}}\PY{l+s+s2}{男性的生还比例:}\PY{l+s+si}{\PYZpc{}f}\PY{l+s+s2}{\PYZdq{}} \PY{o}{\PYZpc{}}\PY{k}{male\PYZus{}survived\PYZus{}rate}
\end{Verbatim}


    \begin{Verbatim}[commandchars=\\\{\}]
女性的生还比例:0.742038
男性的生还比例:0.188908

    \end{Verbatim}

    正如前面图所示,女性的生坏比例比男性高4倍还多。那么还有哪些方面我们可以继续挖掘一些,下面我们将分别从年龄/船舱级别/是否有兄弟姐妹/出发港口分别来看看生还比例。

\paragraph{- 年龄}\label{ux5e74ux9f84}

年龄的数据看,一方面0.42岁到80岁不等,按照这个划分太细,得不出什么结论,因此下面将年龄划分几个部分来看看:
- 0~12岁:儿童; - 13~18岁:青少年,基本上未婚; -
19~59岁,成年,行为完全可以自理,有的可能已经有孩子,具有责任感; -
60岁到80岁:老年,可能有孙子了,可能老伴已经离开了。

    \begin{Verbatim}[commandchars=\\\{\}]
{\color{incolor}In [{\color{incolor}225}]:} \PY{c+c1}{\PYZsh{}先看一下当前年龄相关的统计数据}
          \PY{k}{print} \PY{n}{titanic\PYZus{}clean\PYZus{}df}\PY{o}{.}\PY{n}{describe}\PY{p}{(}\PY{p}{)}\PY{p}{[}\PY{l+s+s1}{\PYZsq{}}\PY{l+s+s1}{Age}\PY{l+s+s1}{\PYZsq{}}\PY{p}{]}
\end{Verbatim}


    \begin{Verbatim}[commandchars=\\\{\}]
count    714.000000
mean      29.699118
std       14.526497
min        0.420000
25\%       20.125000
50\%       28.000000
75\%       38.000000
max       80.000000
Name: Age, dtype: float64

    \end{Verbatim}

    \begin{Verbatim}[commandchars=\\\{\}]
{\color{incolor}In [{\color{incolor}226}]:} \PY{c+c1}{\PYZsh{}\PYZsh{}在开始之前,我们先删除原始数据中没有用的信息,比如PassengerId, Name, Ticket}
          \PY{n}{titanic\PYZus{}clean\PYZus{}df} \PY{o}{=} \PY{n}{gender\PYZus{}submission}\PY{o}{.}\PY{n}{drop}\PY{p}{(}\PY{p}{[}\PY{l+s+s1}{\PYZsq{}}\PY{l+s+s1}{PassengerId}\PY{l+s+s1}{\PYZsq{}}\PY{p}{,}\PY{l+s+s1}{\PYZsq{}}\PY{l+s+s1}{Name}\PY{l+s+s1}{\PYZsq{}}\PY{p}{,}\PY{l+s+s1}{\PYZsq{}}\PY{l+s+s1}{Ticket}\PY{l+s+s1}{\PYZsq{}}\PY{p}{]}\PY{p}{,} \PY{n}{axis}\PY{o}{=}\PY{l+m+mi}{1}\PY{p}{)}
          \PY{n}{total\PYZus{}age\PYZus{}count} \PY{o}{=} \PY{n}{titanic\PYZus{}clean\PYZus{}df}\PY{p}{[}\PY{l+s+s1}{\PYZsq{}}\PY{l+s+s1}{Age}\PY{l+s+s1}{\PYZsq{}}\PY{p}{]}\PY{o}{.}\PY{n}{count}\PY{p}{(}\PY{p}{)}
          \PY{n}{total\PYZus{}age\PYZus{}missed} \PY{o}{=} \PY{n}{titanic\PYZus{}clean\PYZus{}df}\PY{p}{[}\PY{l+s+s1}{\PYZsq{}}\PY{l+s+s1}{Survived}\PY{l+s+s1}{\PYZsq{}}\PY{p}{]}\PY{o}{.}\PY{n}{count}\PY{p}{(}\PY{p}{)}\PY{o}{\PYZhy{}}\PY{n}{total\PYZus{}age\PYZus{}count}
          \PY{k}{print} \PY{l+s+s2}{\PYZdq{}}\PY{l+s+s2}{有年龄纪录的数目是}\PY{l+s+si}{\PYZpc{}d}\PY{l+s+s2}{, 丢失了}\PY{l+s+si}{\PYZpc{}d}\PY{l+s+s2}{\PYZdq{}} \PY{o}{\PYZpc{}}\PY{p}{(}\PY{n}{total\PYZus{}age\PYZus{}count}\PY{p}{,}\PY{n}{total\PYZus{}age\PYZus{}missed}\PY{p}{)}
          
          \PY{c+c1}{\PYZsh{}\PYZsh{} 按照上面的划分对数据进行初步加工,每个年龄阶段的所占比例,总数}
          \PY{k}{def} \PY{n+nf}{change\PYZus{}age\PYZus{}category}\PY{p}{(}\PY{n}{age}\PY{p}{)}\PY{p}{:}
              \PY{k}{if} \PY{n}{age}\PY{o}{\PYZlt{}}\PY{o}{=}\PY{l+m+mi}{12}\PY{p}{:}
                  \PY{k}{return} \PY{l+s+s1}{\PYZsq{}}\PY{l+s+s1}{Child}\PY{l+s+s1}{\PYZsq{}}
              \PY{k}{elif} \PY{p}{(}\PY{n}{age} \PY{o}{\PYZlt{}}\PY{o}{=} \PY{l+m+mi}{18} \PY{o+ow}{and} \PY{n}{age} \PY{o}{\PYZgt{}} \PY{l+m+mi}{12}\PY{p}{)}\PY{p}{:}
                  \PY{k}{return} \PY{l+s+s1}{\PYZsq{}}\PY{l+s+s1}{Teen}\PY{l+s+s1}{\PYZsq{}}
              \PY{k}{elif} \PY{p}{(}\PY{n}{age} \PY{o}{\PYZlt{}}\PY{o}{=} \PY{l+m+mi}{60} \PY{o+ow}{and} \PY{n}{age} \PY{o}{\PYZgt{}} \PY{l+m+mi}{18}\PY{p}{)}\PY{p}{:}
                  \PY{k}{return} \PY{l+s+s1}{\PYZsq{}}\PY{l+s+s1}{Adult}\PY{l+s+s1}{\PYZsq{}}
              \PY{k}{elif} \PY{p}{(}\PY{n}{age}\PY{o}{\PYZgt{}}\PY{l+m+mi}{60}\PY{p}{)}\PY{p}{:}
                  \PY{k}{return} \PY{l+s+s1}{\PYZsq{}}\PY{l+s+s1}{Senior}\PY{l+s+s1}{\PYZsq{}}
          \PY{c+c1}{\PYZsh{}\PYZsh{} 增加Age\PYZus{}level字段}
          \PY{n}{titanic\PYZus{}clean\PYZus{}df}\PY{p}{[}\PY{l+s+s1}{\PYZsq{}}\PY{l+s+s1}{Age\PYZus{}level}\PY{l+s+s1}{\PYZsq{}}\PY{p}{]} \PY{o}{=}  \PY{n}{titanic\PYZus{}clean\PYZus{}df}\PY{p}{[}\PY{l+s+s1}{\PYZsq{}}\PY{l+s+s1}{Age}\PY{l+s+s1}{\PYZsq{}}\PY{p}{]}\PY{o}{.}\PY{n}{apply}\PY{p}{(}\PY{n}{change\PYZus{}age\PYZus{}category}\PY{p}{)}
          
          \PY{c+c1}{\PYZsh{}\PYZsh{} 画出不同年龄段人员的存活比例。}
          \PY{n}{by\PYZus{}age} \PY{o}{=} \PY{n}{titanic\PYZus{}clean\PYZus{}df}\PY{o}{.}\PY{n}{groupby}\PY{p}{(}\PY{l+s+s1}{\PYZsq{}}\PY{l+s+s1}{Age\PYZus{}level}\PY{l+s+s1}{\PYZsq{}}\PY{p}{)}\PY{p}{[}\PY{l+s+s1}{\PYZsq{}}\PY{l+s+s1}{Survived}\PY{l+s+s1}{\PYZsq{}}\PY{p}{]}\PY{o}{.}\PY{n}{mean}\PY{p}{(}\PY{p}{)}
          \PY{k}{print} \PY{n}{by\PYZus{}age}
          
          \PY{n}{fig}\PY{p}{,} \PY{p}{(}\PY{n}{axis1}\PY{p}{,}\PY{n}{axis2}\PY{p}{)} \PY{o}{=} \PY{n}{plt}\PY{o}{.}\PY{n}{subplots}\PY{p}{(}\PY{l+m+mi}{1}\PY{p}{,}\PY{l+m+mi}{2}\PY{p}{,}\PY{n}{figsize}\PY{o}{=}\PY{p}{(}\PY{l+m+mi}{10}\PY{p}{,}\PY{l+m+mi}{5}\PY{p}{)}\PY{p}{)} 
          \PY{n}{by\PYZus{}age}\PY{o}{.}\PY{n}{plot}\PY{p}{(}\PY{n}{kind}\PY{o}{=}\PY{l+s+s1}{\PYZsq{}}\PY{l+s+s1}{bar}\PY{l+s+s1}{\PYZsq{}}\PY{p}{,} \PY{n}{ax}\PY{o}{=}\PY{n}{axis1}\PY{p}{)}
          
          \PY{c+c1}{\PYZsh{}\PYZsh{} 画出不同年龄段人员的存活和死亡对比,更加有参照性}
          \PY{n}{sns}\PY{o}{.}\PY{n}{countplot}\PY{p}{(}\PY{l+s+s1}{\PYZsq{}}\PY{l+s+s1}{Age\PYZus{}level}\PY{l+s+s1}{\PYZsq{}}\PY{p}{,}\PY{n}{data}\PY{o}{=}\PY{n}{titanic\PYZus{}clean\PYZus{}df}\PY{p}{,}\PY{n}{hue}\PY{o}{=}\PY{l+s+s2}{\PYZdq{}}\PY{l+s+s2}{Survived}\PY{l+s+s2}{\PYZdq{}}\PY{p}{,}\PY{n}{ax}\PY{o}{=}\PY{n}{axis2}\PY{p}{)}
\end{Verbatim}


    \begin{Verbatim}[commandchars=\\\{\}]
有年龄纪录的数目是714, 丢失了177
Age\_level
Adult     0.388788
Child     0.579710
Senior    0.227273
Teen      0.428571
Name: Survived, dtype: float64

    \end{Verbatim}

\begin{Verbatim}[commandchars=\\\{\}]
{\color{outcolor}Out[{\color{outcolor}226}]:} <matplotlib.axes.\_subplots.AxesSubplot at 0x1a1c1e3d50>
\end{Verbatim}
            
    \begin{center}
    \adjustimage{max size={0.9\linewidth}{0.9\paperheight}}{output_13_2.png}
    \end{center}
    { \hspace*{\fill} \\}
    
    观察: - 从上图可以看到,child存活下来的人多于死于灾难的人; -
Teen虽然存活下来的人数少一些,也差不多很接近; -
Adult存活下来的人要比灾难中死去的人少将近1/3; -
老人的存活比例很低,可以说只有1/3不到。

2)上图是有年龄的人员在714个记录下的情况,如果我们把没有年龄信息的人员补充进来,可以再看看这个存活比例。补充年龄的方法有好几种,都不是很完美:
-
按照以上划分求出不同年龄段的比例,假定人员年龄丢失的信息是按照这个比例,但是也有可能不是按照这个比例;
- 按照当前均值来在一定范围补充随机值,原则上不改变最终年龄的均值。

为了方便起见我们采用了第二种方式来补充年龄数据。

    \begin{Verbatim}[commandchars=\\\{\}]
{\color{incolor}In [{\color{incolor}227}]:} \PY{c+c1}{\PYZsh{}\PYZsh{} 补充177个随机年龄到该矩阵来}
          \PY{c+c1}{\PYZsh{}\PYZsh{} 求年龄随机数,范围在 (mean \PYZhy{} 2std, mean + 2std)}
          \PY{n}{average\PYZus{}age\PYZus{}titanic} \PY{o}{=} \PY{n}{titanic\PYZus{}clean\PYZus{}df}\PY{o}{.}\PY{n}{describe}\PY{p}{(}\PY{p}{)}\PY{p}{[}\PY{l+s+s1}{\PYZsq{}}\PY{l+s+s1}{Age}\PY{l+s+s1}{\PYZsq{}}\PY{p}{]}\PY{p}{[}\PY{l+m+mi}{1}\PY{p}{]}
          \PY{n}{std\PYZus{}age\PYZus{}titanic} \PY{o}{=} \PY{n}{titanic\PYZus{}clean\PYZus{}df}\PY{o}{.}\PY{n}{describe}\PY{p}{(}\PY{p}{)}\PY{p}{[}\PY{l+s+s1}{\PYZsq{}}\PY{l+s+s1}{Age}\PY{l+s+s1}{\PYZsq{}}\PY{p}{]}\PY{p}{[}\PY{l+m+mi}{2}\PY{p}{]}
          \PY{n}{rand\PYZus{}age} \PY{o}{=} \PY{n}{np}\PY{o}{.}\PY{n}{random}\PY{o}{.}\PY{n}{randint}\PY{p}{(}\PY{n}{average\PYZus{}age\PYZus{}titanic} \PY{o}{\PYZhy{}} \PY{l+m+mi}{2}\PY{o}{*}\PY{n}{std\PYZus{}age\PYZus{}titanic}\PY{p}{,} \PY{n}{average\PYZus{}age\PYZus{}titanic} \PY{o}{+} \PY{l+m+mi}{2}\PY{o}{*}\PY{n}{std\PYZus{}age\PYZus{}titanic}\PY{p}{,} \PY{n}{size} \PY{o}{=} \PY{n}{total\PYZus{}age\PYZus{}missed}\PY{p}{)}
          
          \PY{c+c1}{\PYZsh{}\PYZsh{} 把NAN替换}
          \PY{n}{titanic\PYZus{}clean\PYZus{}df}\PY{p}{[}\PY{l+s+s2}{\PYZdq{}}\PY{l+s+s2}{Age}\PY{l+s+s2}{\PYZdq{}}\PY{p}{]}\PY{p}{[}\PY{n}{np}\PY{o}{.}\PY{n}{isnan}\PY{p}{(}\PY{n}{titanic\PYZus{}clean\PYZus{}df}\PY{p}{[}\PY{l+s+s2}{\PYZdq{}}\PY{l+s+s2}{Age}\PY{l+s+s2}{\PYZdq{}}\PY{p}{]}\PY{p}{)}\PY{p}{]} \PY{o}{=} \PY{n}{rand\PYZus{}age}
          
          \PY{c+c1}{\PYZsh{}\PYZsh{} 之后我们再看一下统计数据}
          \PY{k}{print} \PY{n}{titanic\PYZus{}clean\PYZus{}df}\PY{o}{.}\PY{n}{describe}\PY{p}{(}\PY{p}{)}\PY{p}{[}\PY{l+s+s1}{\PYZsq{}}\PY{l+s+s1}{Age}\PY{l+s+s1}{\PYZsq{}}\PY{p}{]}
\end{Verbatim}


    \begin{Verbatim}[commandchars=\\\{\}]
count    891.000000
mean      29.773479
std       14.812861
min        0.000000
25\%       20.000000
50\%       28.000000
75\%       39.000000
max       80.000000
Name: Age, dtype: float64

    \end{Verbatim}

    \begin{Verbatim}[commandchars=\\\{\}]
/Users/gaochuang/anaconda2/lib/python2.7/site-packages/ipykernel\_launcher.py:8: SettingWithCopyWarning: 
A value is trying to be set on a copy of a slice from a DataFrame

See the caveats in the documentation: http://pandas.pydata.org/pandas-docs/stable/indexing.html\#indexing-view-versus-copy
  

    \end{Verbatim}

    确实的年龄数据补全了,平均值和方差没有太大变化,这时候我们再来看一下不同年龄阶段的生存率,是否有不一样的结果?

    \begin{Verbatim}[commandchars=\\\{\}]
{\color{incolor}In [{\color{incolor}228}]:} \PY{c+c1}{\PYZsh{}\PYZsh{} 对添加missed人进行人群划分}
          \PY{k}{def} \PY{n+nf}{add\PYZus{}missed\PYZus{}age\PYZus{}category}\PY{p}{(}\PY{n}{passenger}\PY{p}{)}\PY{p}{:}
              \PY{n}{age}\PY{p}{,} \PY{n}{age\PYZus{}level} \PY{o}{=} \PY{n}{passenger}
              \PY{k}{if} \PY{n}{age\PYZus{}level} \PY{o}{==}\PY{l+s+s2}{\PYZdq{}}\PY{l+s+s2}{\PYZdq{}}\PY{p}{:}
                  \PY{n}{age\PYZus{}level} \PY{o}{=} \PY{n}{change\PYZus{}age\PYZus{}category}\PY{p}{(}\PY{n}{age}\PY{p}{)}
          
          \PY{n}{titanic\PYZus{}clean\PYZus{}df}\PY{p}{[}\PY{p}{[}\PY{l+s+s2}{\PYZdq{}}\PY{l+s+s2}{Age}\PY{l+s+s2}{\PYZdq{}}\PY{p}{,}\PY{l+s+s2}{\PYZdq{}}\PY{l+s+s2}{Age\PYZus{}level}\PY{l+s+s2}{\PYZdq{}}\PY{p}{]}\PY{p}{]}\PY{o}{.}\PY{n}{apply}\PY{p}{(}\PY{n}{add\PYZus{}missed\PYZus{}age\PYZus{}category}\PY{p}{,} \PY{n}{axis}\PY{o}{=}\PY{l+m+mi}{1}\PY{p}{)}
          \PY{n}{by\PYZus{}age\PYZus{}new} \PY{o}{=} \PY{n}{titanic\PYZus{}clean\PYZus{}df}\PY{o}{.}\PY{n}{groupby}\PY{p}{(}\PY{l+s+s1}{\PYZsq{}}\PY{l+s+s1}{Age\PYZus{}level}\PY{l+s+s1}{\PYZsq{}}\PY{p}{)}\PY{p}{[}\PY{l+s+s1}{\PYZsq{}}\PY{l+s+s1}{Survived}\PY{l+s+s1}{\PYZsq{}}\PY{p}{]}\PY{o}{.}\PY{n}{mean}\PY{p}{(}\PY{p}{)}
          \PY{k}{print} \PY{n}{by\PYZus{}age\PYZus{}new}
          
          \PY{c+c1}{\PYZsh{}\PYZsh{} 画出不同年龄段人员的存活比例}
          \PY{n}{fig}\PY{p}{,} \PY{p}{(}\PY{n}{axis1}\PY{p}{,}\PY{n}{axis2}\PY{p}{)} \PY{o}{=} \PY{n}{plt}\PY{o}{.}\PY{n}{subplots}\PY{p}{(}\PY{l+m+mi}{1}\PY{p}{,}\PY{l+m+mi}{2}\PY{p}{,}\PY{n}{figsize}\PY{o}{=}\PY{p}{(}\PY{l+m+mi}{10}\PY{p}{,}\PY{l+m+mi}{5}\PY{p}{)}\PY{p}{)} 
          \PY{n}{by\PYZus{}age}\PY{o}{.}\PY{n}{plot}\PY{p}{(}\PY{n}{kind} \PY{o}{=} \PY{l+s+s2}{\PYZdq{}}\PY{l+s+s2}{bar}\PY{l+s+s2}{\PYZdq{}}\PY{p}{,} \PY{n}{ax}\PY{o}{=}\PY{n}{axis1}\PY{p}{)}
          
          \PY{c+c1}{\PYZsh{}\PYZsh{} 画出不同年龄段人员的存活和死亡对比,更加有参照性}
          \PY{n}{sns}\PY{o}{.}\PY{n}{countplot}\PY{p}{(}\PY{n}{x}\PY{o}{=}\PY{l+s+s1}{\PYZsq{}}\PY{l+s+s1}{Age\PYZus{}level}\PY{l+s+s1}{\PYZsq{}}\PY{p}{,}\PY{n}{data}\PY{o}{=}\PY{n}{titanic\PYZus{}clean\PYZus{}df}\PY{p}{,} \PY{n}{hue}\PY{o}{=}\PY{l+s+s2}{\PYZdq{}}\PY{l+s+s2}{Survived}\PY{l+s+s2}{\PYZdq{}}\PY{p}{,} \PY{n}{ax}\PY{o}{=}\PY{n}{axis2}\PY{p}{)}
\end{Verbatim}


    \begin{Verbatim}[commandchars=\\\{\}]
Age\_level
Adult     0.388788
Child     0.579710
Senior    0.227273
Teen      0.428571
Name: Survived, dtype: float64

    \end{Verbatim}

\begin{Verbatim}[commandchars=\\\{\}]
{\color{outcolor}Out[{\color{outcolor}228}]:} <matplotlib.axes.\_subplots.AxesSubplot at 0x1a1b6ef8d0>
\end{Verbatim}
            
    \begin{center}
    \adjustimage{max size={0.9\linewidth}{0.9\paperheight}}{output_17_2.png}
    \end{center}
    { \hspace*{\fill} \\}
    
    \begin{Verbatim}[commandchars=\\\{\}]
{\color{incolor}In [{\color{incolor}229}]:} \PY{c+c1}{\PYZsh{}\PYZsh{} 最终打出不同年龄段的生还率}
          \PY{k}{print} \PY{l+s+s2}{\PYZdq{}}\PY{l+s+s2}{Child(0~12)的存活率是 }\PY{l+s+si}{\PYZpc{}f}\PY{l+s+s2}{\PYZdq{}} \PY{o}{\PYZpc{}}\PY{p}{(}\PY{n}{by\PYZus{}age\PYZus{}new}\PY{p}{[}\PY{l+s+s1}{\PYZsq{}}\PY{l+s+s1}{Child}\PY{l+s+s1}{\PYZsq{}}\PY{p}{]}\PY{p}{)}
          \PY{k}{print} \PY{l+s+s2}{\PYZdq{}}\PY{l+s+s2}{Teen (12\PYZti{}18) 的存活率是 }\PY{l+s+si}{\PYZpc{}f}\PY{l+s+s2}{\PYZdq{}} \PY{o}{\PYZpc{}}\PY{p}{(}\PY{n}{by\PYZus{}age\PYZus{}new}\PY{p}{[}\PY{l+s+s1}{\PYZsq{}}\PY{l+s+s1}{Teen}\PY{l+s+s1}{\PYZsq{}}\PY{p}{]}\PY{p}{)}
          \PY{k}{print} \PY{l+s+s2}{\PYZdq{}}\PY{l+s+s2}{Adult (19\PYZti{}59) 的存活率是 }\PY{l+s+si}{\PYZpc{}f}\PY{l+s+s2}{\PYZdq{}} \PY{o}{\PYZpc{}}\PY{p}{(}\PY{n}{by\PYZus{}age\PYZus{}new}\PY{p}{[}\PY{l+s+s1}{\PYZsq{}}\PY{l+s+s1}{Adult}\PY{l+s+s1}{\PYZsq{}}\PY{p}{]}\PY{p}{)}
          \PY{k}{print} \PY{l+s+s2}{\PYZdq{}}\PY{l+s+s2}{Senior (60岁以上) 的存活率是 }\PY{l+s+si}{\PYZpc{}f}\PY{l+s+s2}{\PYZdq{}} \PY{o}{\PYZpc{}}\PY{p}{(}\PY{n}{by\PYZus{}age\PYZus{}new}\PY{p}{[}\PY{l+s+s1}{\PYZsq{}}\PY{l+s+s1}{Senior}\PY{l+s+s1}{\PYZsq{}}\PY{p}{]}\PY{p}{)}
\end{Verbatim}


    \begin{Verbatim}[commandchars=\\\{\}]
Child(0~12)的存活率是 0.579710
Teen (12\textasciitilde{}18) 的存活率是 0.428571
Adult (19\textasciitilde{}59) 的存活率是 0.388788
Senior (60岁以上) 的存活率是 0.227273

    \end{Verbatim}

    补充完数据再看看每个年龄段的存活率,基本没有改变,Child的存活率最高,超过50\%,接下来是Teen接近50\%,成年人的存活率排第三,是38.9\%,老年人的存活率最小,才22.7\%。

因此我们可以猜测,在灾难发生的时候,小孩获得了更多的生还机会,受到了大家的保护,有些成年人把自己的生还机会让给了孩子。
60岁上的老人可能是求生欲望不强,在灾难发生后无论是体力还是心力都不是很强,存活的几率不高。

    \paragraph{-
船舱等级的分析}\label{ux8239ux8231ux7b49ux7ea7ux7684ux5206ux6790}

    \begin{Verbatim}[commandchars=\\\{\}]
{\color{incolor}In [{\color{incolor}230}]:} \PY{c+c1}{\PYZsh{}先看一下当前船舱相关的统计数据}
          \PY{k}{print} \PY{n}{titanic\PYZus{}clean\PYZus{}df}\PY{p}{[}\PY{l+s+s1}{\PYZsq{}}\PY{l+s+s1}{Pclass}\PY{l+s+s1}{\PYZsq{}}\PY{p}{]}\PY{o}{.}\PY{n}{value\PYZus{}counts}\PY{p}{(}\PY{p}{)}
          \PY{n}{by\PYZus{}pclass} \PY{o}{=} \PY{n}{titanic\PYZus{}clean\PYZus{}df}\PY{o}{.}\PY{n}{groupby}\PY{p}{(}\PY{l+s+s1}{\PYZsq{}}\PY{l+s+s1}{Pclass}\PY{l+s+s1}{\PYZsq{}}\PY{p}{)}\PY{p}{[}\PY{l+s+s1}{\PYZsq{}}\PY{l+s+s1}{Survived}\PY{l+s+s1}{\PYZsq{}}\PY{p}{]}\PY{o}{.}\PY{n}{mean}\PY{p}{(}\PY{p}{)}
          \PY{k}{print} \PY{n}{by\PYZus{}pclass}
\end{Verbatim}


    \begin{Verbatim}[commandchars=\\\{\}]
3    491
1    216
2    184
Name: Pclass, dtype: int64
Pclass
1    0.629630
2    0.472826
3    0.242363
Name: Survived, dtype: float64

    \end{Verbatim}

    \begin{Verbatim}[commandchars=\\\{\}]
{\color{incolor}In [{\color{incolor}231}]:} \PY{c+c1}{\PYZsh{}\PYZsh{} 我们来画图,看看不同级别的生存数目和死亡人数之间的对比}
          \PY{n}{sns}\PY{o}{.}\PY{n}{factorplot}\PY{p}{(}\PY{l+s+s1}{\PYZsq{}}\PY{l+s+s1}{Pclass}\PY{l+s+s1}{\PYZsq{}}\PY{p}{,}\PY{n}{data}\PY{o}{=}\PY{n}{titanic\PYZus{}clean\PYZus{}df}\PY{p}{,}\PY{n}{kind}\PY{o}{=}\PY{l+s+s2}{\PYZdq{}}\PY{l+s+s2}{count}\PY{l+s+s2}{\PYZdq{}}\PY{p}{,}\PY{n}{hue}\PY{o}{=}\PY{l+s+s2}{\PYZdq{}}\PY{l+s+s2}{Survived}\PY{l+s+s2}{\PYZdq{}}\PY{p}{)} 
          
          \PY{c+c1}{\PYZsh{}\PYZsh{} 打印出不同船舱的存活率}
          \PY{k}{print} \PY{l+s+s2}{\PYZdq{}}\PY{l+s+s2}{1级船舱的存活率是: }\PY{l+s+si}{\PYZpc{}f}\PY{l+s+s2}{\PYZdq{}} \PY{o}{\PYZpc{}}\PY{p}{(}\PY{n}{by\PYZus{}pclass}\PY{p}{[}\PY{l+m+mi}{1}\PY{p}{]}\PY{p}{)}
          \PY{k}{print} \PY{l+s+s2}{\PYZdq{}}\PY{l+s+s2}{2级船舱的存活率是: }\PY{l+s+si}{\PYZpc{}f}\PY{l+s+s2}{\PYZdq{}} \PY{o}{\PYZpc{}}\PY{p}{(}\PY{n}{by\PYZus{}pclass}\PY{p}{[}\PY{l+m+mi}{2}\PY{p}{]}\PY{p}{)}
          \PY{k}{print} \PY{l+s+s2}{\PYZdq{}}\PY{l+s+s2}{3级船舱的存活率是: }\PY{l+s+si}{\PYZpc{}f}\PY{l+s+s2}{\PYZdq{}} \PY{o}{\PYZpc{}}\PY{p}{(}\PY{n}{by\PYZus{}pclass}\PY{p}{[}\PY{l+m+mi}{3}\PY{p}{]}\PY{p}{)}
\end{Verbatim}


    \begin{Verbatim}[commandchars=\\\{\}]
1级船舱的存活率是: 0.629630
2级船舱的存活率是: 0.472826
3级船舱的存活率是: 0.242363

    \end{Verbatim}

    \begin{center}
    \adjustimage{max size={0.9\linewidth}{0.9\paperheight}}{output_22_1.png}
    \end{center}
    { \hspace*{\fill} \\}
    
    观察:
无论从数据上还是看图,我们都发现1级船舱的存活率非常高,为63\%,二级船舱的存活率次之,也有47\%,三级船舱存活率才24\%,非常低。
这也说明有钱人还是收到了更多照顾,也有可能一级船舱在船的上层,发生灾难的时候他们更容易逃离,即使排队获得救生艇的机会比下层船舱的乘客。

    \paragraph{- 兄弟姐妹}\label{ux5144ux5f1fux59d0ux59b9}

    \begin{Verbatim}[commandchars=\\\{\}]
{\color{incolor}In [{\color{incolor}232}]:} \PY{c+c1}{\PYZsh{}先看一下当前兄弟姐妹相关的统计数据}
          \PY{k}{print} \PY{n}{titanic\PYZus{}clean\PYZus{}df}\PY{p}{[}\PY{l+s+s1}{\PYZsq{}}\PY{l+s+s1}{SibSp}\PY{l+s+s1}{\PYZsq{}}\PY{p}{]}\PY{o}{.}\PY{n}{value\PYZus{}counts}\PY{p}{(}\PY{p}{)}
          \PY{n}{by\PYZus{}SibSp} \PY{o}{=} \PY{n}{titanic\PYZus{}clean\PYZus{}df}\PY{o}{.}\PY{n}{groupby}\PY{p}{(}\PY{l+s+s1}{\PYZsq{}}\PY{l+s+s1}{SibSp}\PY{l+s+s1}{\PYZsq{}}\PY{p}{)}\PY{p}{[}\PY{l+s+s1}{\PYZsq{}}\PY{l+s+s1}{Survived}\PY{l+s+s1}{\PYZsq{}}\PY{p}{]}\PY{o}{.}\PY{n}{mean}\PY{p}{(}\PY{p}{)}
          \PY{k}{print} \PY{n}{by\PYZus{}SibSp}
\end{Verbatim}


    \begin{Verbatim}[commandchars=\\\{\}]
0    608
1    209
2     28
4     18
3     16
8      7
5      5
Name: SibSp, dtype: int64
SibSp
0    0.345395
1    0.535885
2    0.464286
3    0.250000
4    0.166667
5    0.000000
8    0.000000
Name: Survived, dtype: float64

    \end{Verbatim}

    从数据上看,孤独一个人的乘客占大多数,有608个;而有1个次之,209个,或者2个兄弟姐妹的人为28个,逐渐越来越少。
从存活率上来看有一个兄弟姐们在船上的人存活率最高,超过一半,为53.59\%;
有2个兄弟姐们在船上的人存活率接近一半,为第二,为46.42\%;
人数最多的,孤独一个人的乘客只有1/3多一点,为34.53\%; 其他都比较差。
我们来画图,这样更直观一些。

    \begin{Verbatim}[commandchars=\\\{\}]
{\color{incolor}In [{\color{incolor}233}]:} \PY{c+c1}{\PYZsh{}\PYZsh{} 先画出根据兄弟姐们个数分类的人数对比柱状图}
          \PY{n}{fig}\PY{p}{,} \PY{p}{(}\PY{n}{axis1}\PY{p}{,}\PY{n}{axis2}\PY{p}{,}\PY{n}{axis3}\PY{p}{)} \PY{o}{=} \PY{n}{plt}\PY{o}{.}\PY{n}{subplots}\PY{p}{(}\PY{l+m+mi}{1}\PY{p}{,}\PY{l+m+mi}{3}\PY{p}{,}\PY{n}{figsize}\PY{o}{=}\PY{p}{(}\PY{l+m+mi}{15}\PY{p}{,}\PY{l+m+mi}{5}\PY{p}{)}\PY{p}{)} 
          \PY{n}{sns}\PY{o}{.}\PY{n}{countplot}\PY{p}{(}\PY{n}{x}\PY{o}{=}\PY{l+s+s1}{\PYZsq{}}\PY{l+s+s1}{SibSp}\PY{l+s+s1}{\PYZsq{}}\PY{p}{,}\PY{n}{data}\PY{o}{=}\PY{n}{titanic\PYZus{}clean\PYZus{}df}\PY{p}{,}\PY{n}{ax}\PY{o}{=}\PY{n}{axis1}\PY{p}{)}
          \PY{n}{plt}\PY{o}{.}\PY{n}{title}\PY{p}{(}\PY{l+s+s1}{\PYZsq{}}\PY{l+s+s1}{Total counts by siblings}\PY{l+s+s1}{\PYZsq{}}\PY{p}{)}
          
          \PY{c+c1}{\PYZsh{}\PYZsh{} 再画出根据兄弟姐们个数分类的总人数和存活人数对比图}
          \PY{n}{sns}\PY{o}{.}\PY{n}{countplot}\PY{p}{(}\PY{n}{x}\PY{o}{=}\PY{l+s+s1}{\PYZsq{}}\PY{l+s+s1}{SibSp}\PY{l+s+s1}{\PYZsq{}}\PY{p}{,}\PY{n}{data}\PY{o}{=}\PY{n}{titanic\PYZus{}clean\PYZus{}df}\PY{p}{,}\PY{n}{hue}\PY{o}{=}\PY{l+s+s2}{\PYZdq{}}\PY{l+s+s2}{Survived}\PY{l+s+s2}{\PYZdq{}}\PY{p}{,}\PY{n}{ax}\PY{o}{=}\PY{n}{axis2}\PY{p}{)}
          \PY{n}{plt}\PY{o}{.}\PY{n}{title}\PY{p}{(}\PY{l+s+s1}{\PYZsq{}}\PY{l+s+s1}{Total counts Vs Survived by siblings}\PY{l+s+s1}{\PYZsq{}}\PY{p}{)}
          
          \PY{c+c1}{\PYZsh{}\PYZsh{} 最后我们画出根据兄弟姐们个数分类后的存活率}
          \PY{n}{by\PYZus{}SibSp}\PY{o}{.}\PY{n}{plot}\PY{p}{(}\PY{n}{kind} \PY{o}{=} \PY{l+s+s2}{\PYZdq{}}\PY{l+s+s2}{bar}\PY{l+s+s2}{\PYZdq{}}\PY{p}{,}\PY{n}{ax}\PY{o}{=}\PY{n}{axis3}\PY{p}{)}
          \PY{n}{plt}\PY{o}{.}\PY{n}{title}\PY{p}{(}\PY{l+s+s1}{\PYZsq{}}\PY{l+s+s1}{The Survived rate by Siblings}\PY{l+s+s1}{\PYZsq{}}\PY{p}{)}
\end{Verbatim}


\begin{Verbatim}[commandchars=\\\{\}]
{\color{outcolor}Out[{\color{outcolor}233}]:} Text(0.5,1,u'The Survived rate by Siblings')
\end{Verbatim}
            
    \begin{center}
    \adjustimage{max size={0.9\linewidth}{0.9\paperheight}}{output_27_1.png}
    \end{center}
    { \hspace*{\fill} \\}
    
    \paragraph{-
父母或者孩子在船舱个数}\label{ux7236ux6bcdux6216ux8005ux5b69ux5b50ux5728ux8239ux8231ux4e2aux6570}

    \begin{Verbatim}[commandchars=\\\{\}]
{\color{incolor}In [{\color{incolor}234}]:} \PY{c+c1}{\PYZsh{}先看一下当前父母孩子们个数相关的统计数据}
          \PY{k}{print} \PY{n}{titanic\PYZus{}clean\PYZus{}df}\PY{p}{[}\PY{l+s+s1}{\PYZsq{}}\PY{l+s+s1}{Parch}\PY{l+s+s1}{\PYZsq{}}\PY{p}{]}\PY{o}{.}\PY{n}{value\PYZus{}counts}\PY{p}{(}\PY{p}{)}
          \PY{n}{by\PYZus{}Parch} \PY{o}{=} \PY{n}{titanic\PYZus{}clean\PYZus{}df}\PY{o}{.}\PY{n}{groupby}\PY{p}{(}\PY{l+s+s1}{\PYZsq{}}\PY{l+s+s1}{Parch}\PY{l+s+s1}{\PYZsq{}}\PY{p}{)}\PY{p}{[}\PY{l+s+s1}{\PYZsq{}}\PY{l+s+s1}{Survived}\PY{l+s+s1}{\PYZsq{}}\PY{p}{]}\PY{o}{.}\PY{n}{mean}\PY{p}{(}\PY{p}{)}
          \PY{k}{print} \PY{n}{by\PYZus{}Parch}
\end{Verbatim}


    \begin{Verbatim}[commandchars=\\\{\}]
0    678
1    118
2     80
5      5
3      5
4      4
6      1
Name: Parch, dtype: int64
Parch
0    0.343658
1    0.550847
2    0.500000
3    0.600000
4    0.000000
5    0.200000
6    0.000000
Name: Survived, dtype: float64

    \end{Verbatim}

    \textbf{观察:} 1. 人员占比 -
没有父母或者孩子在船舱的人最多,678个和上面兄弟姐们的数据几乎是一致的,占76\%;
-
其次是有一个在船上,或者父子两人,父女两人,或者母子两人,或者母女两人,他们差不多有118人,占总人数的13\%
- 有两个在船上80人,占9.0\%左右; -
剩下加起来1\%~2\%,从统计学的角度看,样本量太少,可以忽略不计。 2.
存活率比较: -
有3个孩子或者父母孩子加起来是4个的在船上的存活率最高,差不多60\%,但是样本量太少,先暂时忽略;
- 有一个或者2个存活率很接近,都超过了50\%,这个数据还是很能说明问题; -
没有孩子在船上的,存活率很低,1/3强,34.36\%。
因此可以看到有孩子的比没有孩子在船上的存活率要高,是不是因为孩子在船上,必须要想办法寻找各种办法让孩子活下去,也符合我们之前的发现,孩子的存活率高的观察。
下面我们画图来观察一下。

    \begin{Verbatim}[commandchars=\\\{\}]
{\color{incolor}In [{\color{incolor}235}]:} \PY{c+c1}{\PYZsh{}\PYZsh{} 先画出根据父母孩子们个数分类的人数对比柱状图}
          \PY{n}{fig}\PY{p}{,} \PY{p}{(}\PY{n}{axis1}\PY{p}{,}\PY{n}{axis2}\PY{p}{,}\PY{n}{axis3}\PY{p}{)} \PY{o}{=} \PY{n}{plt}\PY{o}{.}\PY{n}{subplots}\PY{p}{(}\PY{l+m+mi}{1}\PY{p}{,}\PY{l+m+mi}{3}\PY{p}{,}\PY{n}{figsize}\PY{o}{=}\PY{p}{(}\PY{l+m+mi}{15}\PY{p}{,}\PY{l+m+mi}{5}\PY{p}{)}\PY{p}{)} 
          \PY{n}{sns}\PY{o}{.}\PY{n}{countplot}\PY{p}{(}\PY{n}{x}\PY{o}{=}\PY{l+s+s1}{\PYZsq{}}\PY{l+s+s1}{Parch}\PY{l+s+s1}{\PYZsq{}}\PY{p}{,}\PY{n}{data}\PY{o}{=}\PY{n}{titanic\PYZus{}clean\PYZus{}df}\PY{p}{,}\PY{n}{ax}\PY{o}{=}\PY{n}{axis1}\PY{p}{)}
          
          \PY{c+c1}{\PYZsh{}\PYZsh{} 再画出根据父母孩子们个数分类的总人数和存活人数对比图}
          \PY{n}{sns}\PY{o}{.}\PY{n}{countplot}\PY{p}{(}\PY{n}{x}\PY{o}{=}\PY{l+s+s1}{\PYZsq{}}\PY{l+s+s1}{Parch}\PY{l+s+s1}{\PYZsq{}}\PY{p}{,}\PY{n}{data}\PY{o}{=}\PY{n}{titanic\PYZus{}clean\PYZus{}df}\PY{p}{,}\PY{n}{ax}\PY{o}{=}\PY{n}{axis2}\PY{p}{,}\PY{n}{hue}\PY{o}{=}\PY{l+s+s2}{\PYZdq{}}\PY{l+s+s2}{Survived}\PY{l+s+s2}{\PYZdq{}}\PY{p}{)}
          
          \PY{c+c1}{\PYZsh{}\PYZsh{} 最后我们画出根据父母孩子们个数分类后的存活率}
          \PY{n}{by\PYZus{}Parch}\PY{o}{.}\PY{n}{plot}\PY{p}{(}\PY{n}{kind} \PY{o}{=} \PY{l+s+s2}{\PYZdq{}}\PY{l+s+s2}{bar}\PY{l+s+s2}{\PYZdq{}}\PY{p}{)}
          \PY{n}{plt}\PY{o}{.}\PY{n}{title}\PY{p}{(}\PY{l+s+s1}{\PYZsq{}}\PY{l+s+s1}{The Survived rate by Parents and Children}\PY{l+s+s1}{\PYZsq{}}\PY{p}{)}
          
          \PY{n}{plt}\PY{o}{.}\PY{n}{show}\PY{p}{(}\PY{p}{)}
\end{Verbatim}


    \begin{center}
    \adjustimage{max size={0.9\linewidth}{0.9\paperheight}}{output_31_0.png}
    \end{center}
    { \hspace*{\fill} \\}
    
    \paragraph{- 乘客票务花费}\label{ux4e58ux5ba2ux7968ux52a1ux82b1ux8d39}

    \begin{Verbatim}[commandchars=\\\{\}]
{\color{incolor}In [{\color{incolor}236}]:} \PY{c+c1}{\PYZsh{}先看一下当前父母孩子们个数相关的统计数据}
          \PY{k}{print} \PY{n}{titanic\PYZus{}clean\PYZus{}df}\PY{p}{[}\PY{l+s+s1}{\PYZsq{}}\PY{l+s+s1}{Fare}\PY{l+s+s1}{\PYZsq{}}\PY{p}{]}\PY{o}{.}\PY{n}{describe}\PY{p}{(}\PY{p}{)}
\end{Verbatim}


    \begin{Verbatim}[commandchars=\\\{\}]
count    891.000000
mean      32.204208
std       49.693429
min        0.000000
25\%        7.910400
50\%       14.454200
75\%       31.000000
max      512.329200
Name: Fare, dtype: float64

    \end{Verbatim}

    从统计数据来看,std比mean还大,说明这个数据比较分散,不能用上面的方法进行分类进行分析。其实票务花费应该和船舱级别是挂钩的,我们看看他们之间有什么关系。

    \begin{Verbatim}[commandchars=\\\{\}]
{\color{incolor}In [{\color{incolor}237}]:} \PY{c+c1}{\PYZsh{}\PYZsh{}看看船票费用的分布}
          \PY{n}{plt}\PY{o}{.}\PY{n}{figure}\PY{p}{(}\PY{n}{figsize}\PY{o}{=}\PY{p}{(}\PY{l+m+mi}{10}\PY{p}{,}\PY{l+m+mi}{5}\PY{p}{)}\PY{p}{)}
          \PY{n}{titanic\PYZus{}clean\PYZus{}df}\PY{p}{[}\PY{l+s+s1}{\PYZsq{}}\PY{l+s+s1}{Fare}\PY{l+s+s1}{\PYZsq{}}\PY{p}{]}\PY{o}{.}\PY{n}{hist}\PY{p}{(}\PY{n}{bins} \PY{o}{=} \PY{l+m+mi}{50}\PY{p}{)}
          
          \PY{c+c1}{\PYZsh{}\PYZsh{}画出船票费用和船舱级别的关系}
          \PY{n}{titanic\PYZus{}clean\PYZus{}df}\PY{o}{.}\PY{n}{boxplot}\PY{p}{(}\PY{n}{column}\PY{o}{=}\PY{l+s+s1}{\PYZsq{}}\PY{l+s+s1}{Fare}\PY{l+s+s1}{\PYZsq{}}\PY{p}{,} \PY{n}{by}\PY{o}{=}\PY{l+s+s1}{\PYZsq{}}\PY{l+s+s1}{Pclass}\PY{l+s+s1}{\PYZsq{}}\PY{p}{,} \PY{n}{showfliers}\PY{o}{=}\PY{n+nb+bp}{False}\PY{p}{)}
          \PY{n}{plt}\PY{o}{.}\PY{n}{show}\PY{p}{(}\PY{p}{)}
\end{Verbatim}


    \begin{center}
    \adjustimage{max size={0.9\linewidth}{0.9\paperheight}}{output_35_0.png}
    \end{center}
    { \hspace*{\fill} \\}
    
    \begin{center}
    \adjustimage{max size={0.9\linewidth}{0.9\paperheight}}{output_35_1.png}
    \end{center}
    { \hspace*{\fill} \\}
    
    从上面两张图可以可以看到,很多人套了很少的钱上船,不到20,2级和3级船舱的船票集中在25英镑以下。
另外,目前还看不到票价和存活率有什么关系?下面我们继续挖掘。

    \begin{Verbatim}[commandchars=\\\{\}]
{\color{incolor}In [{\color{incolor}238}]:} \PY{n}{by\PYZus{}Fare} \PY{o}{=} \PY{n}{titanic\PYZus{}clean\PYZus{}df}\PY{o}{.}\PY{n}{groupby}\PY{p}{(}\PY{l+s+s1}{\PYZsq{}}\PY{l+s+s1}{Survived}\PY{l+s+s1}{\PYZsq{}}\PY{p}{)}\PY{p}{[}\PY{l+s+s1}{\PYZsq{}}\PY{l+s+s1}{Fare}\PY{l+s+s1}{\PYZsq{}}\PY{p}{]}\PY{o}{.}\PY{n}{mean}\PY{p}{(}\PY{p}{)}
          \PY{k}{print} \PY{n}{by\PYZus{}Fare}
          
          \PY{c+c1}{\PYZsh{}\PYZsh{}开始画图,先画一下生死两种人的平均船票价格}
          \PY{n}{fig}\PY{p}{,} \PY{p}{(}\PY{n}{axis1}\PY{p}{,}\PY{n}{axis2}\PY{p}{)} \PY{o}{=} \PY{n}{plt}\PY{o}{.}\PY{n}{subplots}\PY{p}{(}\PY{l+m+mi}{1}\PY{p}{,}\PY{l+m+mi}{2}\PY{p}{,}\PY{n}{figsize}\PY{o}{=}\PY{p}{(}\PY{l+m+mi}{10}\PY{p}{,}\PY{l+m+mi}{5}\PY{p}{)}\PY{p}{)} 
          \PY{n}{by\PYZus{}Fare}\PY{o}{.}\PY{n}{plot}\PY{p}{(}\PY{n}{kind}\PY{o}{=}\PY{l+s+s1}{\PYZsq{}}\PY{l+s+s1}{bar}\PY{l+s+s1}{\PYZsq{}}\PY{p}{,} \PY{n}{ax}\PY{o}{=}\PY{n}{axis1}\PY{p}{)}
          
          \PY{c+c1}{\PYZsh{}\PYZsh{} 由于船票价格浮动太大,我们在用统计学的工具来看看,生死两种人大概的船票价格区间}
          \PY{n}{fare\PYZus{}not\PYZus{}survived} \PY{o}{=} \PY{n}{titanic\PYZus{}clean\PYZus{}df}\PY{p}{[}\PY{l+s+s2}{\PYZdq{}}\PY{l+s+s2}{Fare}\PY{l+s+s2}{\PYZdq{}}\PY{p}{]}\PY{p}{[}\PY{n}{titanic\PYZus{}clean\PYZus{}df}\PY{p}{[}\PY{l+s+s2}{\PYZdq{}}\PY{l+s+s2}{Survived}\PY{l+s+s2}{\PYZdq{}}\PY{p}{]} \PY{o}{==} \PY{l+m+mi}{0}\PY{p}{]}
          \PY{n}{fare\PYZus{}survived}     \PY{o}{=} \PY{n}{titanic\PYZus{}clean\PYZus{}df}\PY{p}{[}\PY{l+s+s2}{\PYZdq{}}\PY{l+s+s2}{Fare}\PY{l+s+s2}{\PYZdq{}}\PY{p}{]}\PY{p}{[}\PY{n}{titanic\PYZus{}clean\PYZus{}df}\PY{p}{[}\PY{l+s+s2}{\PYZdq{}}\PY{l+s+s2}{Survived}\PY{l+s+s2}{\PYZdq{}}\PY{p}{]} \PY{o}{==} \PY{l+m+mi}{1}\PY{p}{]}
          
          \PY{n}{avgerage\PYZus{}fare} \PY{o}{=} \PY{n}{pd}\PY{o}{.}\PY{n}{DataFrame}\PY{p}{(}\PY{p}{[}\PY{n}{fare\PYZus{}not\PYZus{}survived}\PY{o}{.}\PY{n}{mean}\PY{p}{(}\PY{p}{)}\PY{p}{,} \PY{n}{fare\PYZus{}survived}\PY{o}{.}\PY{n}{mean}\PY{p}{(}\PY{p}{)}\PY{p}{]}\PY{p}{)}
          \PY{n}{std\PYZus{}fare}      \PY{o}{=} \PY{n}{pd}\PY{o}{.}\PY{n}{DataFrame}\PY{p}{(}\PY{p}{[}\PY{n}{fare\PYZus{}not\PYZus{}survived}\PY{o}{.}\PY{n}{std}\PY{p}{(}\PY{p}{)}\PY{p}{,} \PY{n}{fare\PYZus{}survived}\PY{o}{.}\PY{n}{std}\PY{p}{(}\PY{p}{)}\PY{p}{]}\PY{p}{)}
          \PY{n}{avgerage\PYZus{}fare}\PY{o}{.}\PY{n}{plot}\PY{p}{(}\PY{n}{yerr}\PY{o}{=}\PY{n}{std\PYZus{}fare}\PY{p}{,}\PY{n}{kind}\PY{o}{=}\PY{l+s+s1}{\PYZsq{}}\PY{l+s+s1}{bar}\PY{l+s+s1}{\PYZsq{}}\PY{p}{,}\PY{n}{legend}\PY{o}{=}\PY{n+nb+bp}{False}\PY{p}{,} \PY{n}{ax}\PY{o}{=}\PY{n}{axis2}\PY{p}{)}
          
          \PY{n}{plt}\PY{o}{.}\PY{n}{show}\PY{p}{(}\PY{p}{)}
\end{Verbatim}


    \begin{Verbatim}[commandchars=\\\{\}]
Survived
0    22.117887
1    48.395408
Name: Fare, dtype: float64

    \end{Verbatim}

    \begin{center}
    \adjustimage{max size={0.9\linewidth}{0.9\paperheight}}{output_37_1.png}
    \end{center}
    { \hspace*{\fill} \\}
    
    \textbf{观察:}
无论是从均值还是船票区间来看,存活下来的人的船票价格要高于死去的人,他们的船票价格在0到55之间,而死去的大部分在20以下。如果和船舱级别挂钩的话,由于船票价格低,大部分人在2级和3级船舱,而掏更多钱买票的人基本在1级船舱,他们的存活率因此比较高。

    \subsubsection{分析总结}\label{ux5206ux6790ux603bux7ed3}

总结分为两个部分,分别是本次数据分析得出的规律和对于分析的限制性进行讨论。

\paragraph{- 数据分析总结}\label{ux6570ux636eux5206ux6790ux603bux7ed3}

本次分析主要探寻泰坦尼克号上的生还率和各因素:性别、年龄、客舱等级、兄弟姐妹个数、父母孩子个数、船票价格的关系。

样本数量为 891,海难发生后,生还者还剩 342 人,生还率为:
\textbf{38\%}。

\textbf{性别}: 在891人中,男性共577人,女性314人,男女比例为 65\% 和
35\%。海难发生后,男性存活了109人,女性为233人,男女比例变为 32\% 和
68\%。男性生还率仅为\textbf{19\%},女性生还率为
\textbf{74\%},远远高于男性的
19\%。可见女性比男性在这次事故中更容易生还,表明``女士优先''的原则在本次事故中得到了发扬。

\textbf{年龄}: -
在891人当中,有年龄数据记录总共是714,缺失了177条记录,平均年龄约为 30
岁, 标准差 15 岁。 -
年龄分布太散,从0.42到85岁,因此我们把此数据做了加工,一方面补充众数,一方面重新划分了四个年龄段:小孩(12岁以下),青少年(12岁到18岁),成年(19岁到60岁)和老年(60岁以上);
-
按照此年龄划分,我们发现儿童的成活率最高,为\textbf{58\%},接下来是青少年为\textbf{423\%},老年人最低才\textbf{23\%}
以上我们可以推断出,当事故发生的时候,绝大多数人还是听从船长的指挥和号召:让女人和孩子先走,另外也反映了当时社会的文明程度。

\textbf{船舱等级}: 泰坦尼克号上有三种船舱类型。海难发生前,一等舱有 216
人,二等舱 184 人,三等舱 491 人,分别占总人数的 24\%, 21\%,
55\%。海难发生后,一等舱、二等舱、三等舱的乘客人数变为136、87、119人,分别占总人数的
40\%, 25\%, 35\%。一等舱生还率为 63\%,二等舱为 47\%,三等舱为
24\%。可见客舱等级越高,生还率越高。

\textbf{兄弟姐妹个数}: -
孤独一个人的乘客占大多数,有608个;而有1个次之,209个,或者2个兄弟姐妹的人为28个,逐渐越来越少。
-
从存活率上来看有一个兄弟姐们在船上的人存活率最高,超过一半,为53.59\%;
有2个兄弟姐们在船上的人存活率接近一半,为第二,为46.42\%; -
人数最多的,孤独一个人的乘客只有1/3多一点,为34.53\%; 其他都比较差。
总之,有兄弟姐妹的乘客的生还几率要高于孤独乘客,这里面具体人文原因需要进一步挖掘和探索。

\textbf{父母和子女同乘} -
没有父母或者孩子在船舱的人最多,678个和上面兄弟姐们的孤独数据几乎是一致的,占76\%;
- 但是没有父母子女同船的乘客,仅生还 233 人,生还率为 34\%。 -
而有父母或子女同船的乘客都超过了50\%,尤其是有3个的,都快到了60\%。
因此,有父母或者子女同乘的生还率要远高于孤独乘客。

\textbf{票价} -
一方面票价的分布也比较散,和船舱级别有一定的关联度,2级和3级船舱的票价集中在25以下;
-
另外一方面,生还下来的人他们的船票价格要高于死去的乘客,无论从均值还是区间范围。
因此,可以推测票价和船舱级别挂钩,由于船票价格低,大部分人在2级和3级船舱,而掏更多钱买票的人基本在1级船舱,他们的存活率因此比较高。

\paragraph{分析限制讨论}\label{ux5206ux6790ux9650ux5236ux8ba8ux8bba}

此数据并非全部乘客的数据,据了解,泰坦尼克号上共有乘客 2224
人,而本数据集共有 891 人。 - 如果该数据集是从 2224
人中随机选出,根据中心极限定理,该样本也足够大,分析结果有代表性; -
如果不是随机选出,那么分析结果就不可靠了。

另外此分析忽略了cabin船舱号和离岸港口,姓名的分析,一方面是cabin的数据很少不全,另外一方面这些分析要结合船舱结构、历史和地理知识,由于本人对这部分不是很熟悉,担心解释不了分析结果,其实分析的路子是一样的。
因此可能还有其他因素影响生还情况,还可以继续寻找数据继续探索。

    \subsubsection{参考文献}\label{ux53c2ux8003ux6587ux732e}

\begin{itemize}
\tightlist
\item
  \href{http://seaborn.pydata.org}{seaborn 官方文档}
\item
  \href{http://pandas.pydata.org}{pandas 官方文档}
\item
  \href{https://www.jianshu.com/p/9a5bce0de13f?utm_campaign=maleskine\&utm_content=note\&utm_medium=seo_notes\&utm_source=recommendation}{泰坦尼克号乘客生存分析------用机器学习告诉你,如果你在当时的船上,有多大机率生还?}
\item
  \href{http://blog.51cto.com/youerning/1711371}{python数据分析实战之泰坦尼克号统计}
\end{itemize}


    % Add a bibliography block to the postdoc
    
    
    
    \end{document}
